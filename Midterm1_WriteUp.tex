\documentclass[conference]{IEEEtran}
\IEEEoverridecommandlockouts
% The preceding line is only needed to identify funding in the first footnote. If that is unneeded, please comment it out.
\usepackage{cite}
\usepackage{amsmath,amssymb,amsfonts}
\usepackage{algorithmic}
\usepackage{graphicx}
\usepackage{textcomp}
\usepackage{xcolor}
\def\BibTeX{{\rm B\kern-.05em{\sc i\kern-.025em b}\kern-.08em
    T\kern-.1667em\lower.7ex\hbox{E}\kern-.125emX}}
\begin{document}

\title{CS485G Midterm 1}

\author{\IEEEauthorblockN{Jonathan Moore}
\IEEEauthorblockA{\textit{Dept. of Electrical and Computer Engineering} \\
\textit{University of Kentucky}\\
Lexington, Kentucky \\
jdmo242@uky.edu}
\and
\IEEEauthorblockN{Oluwafeyisayo Afolabi}
\IEEEauthorblockA{\textit{Dept. of Electrical and Computer Engineering} \\
\textit{University of Kentucky)}\\
Lexington, Kentucky \\
oaaf222@uky.edu}
\and
\IEEEauthorblockN{3\textsuperscript{rd} Given Name Surname}
\IEEEauthorblockA{\textit{dept. name of organization (of Aff.)} \\
\textit{name of organization (of Aff.)}\\
City, Country \\
email address}
}

\maketitle

\begin{abstract}
This project was to create an application that used the CoreLocation framework to track
and record the users location.
\end{abstract}

\section{Introduction}
The primary purpose of this tutorial was to gain experience using the CoreLocation
framework by creating an application that tracks and logs the users location. As an
extention of this primary purpose, the tutorial also offered a number of other opportunities
to learn about various frameworks and aspects of swift as well as experience in storyboarding
and working in Xcode. Through the app design process we were able to learn more about
table views using the UIKit table view controller feature. Using MapKit, we 
learned how to embed a map into the application as well as add custom points of 
interest to it. Another useful feature we gained experience with was the Codable
protocal, which provides an easy way to allow the app data to be encoded and decoded
for storage. We also learned how to use various features of the CoreLocation 
framework such as CLLocationCoordinate2D.
\section{Original Function}

\section{Functionality}

\subsection{Original Function}

\subsubsection{Baseline App}

\subsubsection{First Half of Tutorial}

\subsubsection{Second Half of Tutorial}

\subsection{Added Functions}
Overall, our team added three additional features to the application that were not 
originally included in the tutorial. These features focused on adding functionality
that would tie into our final application and that created methods that we could
reuse in our future projects.

\subsubsection{Find Closest Restaurant}
The first additional feature that we added was to allow the user to see the closest
restaurant to them in the map view screen. From a user standpoint, we did this by 
adding a button in the top left of the navigation bar that says "Find Food." When 
pressed, this button will create a pin on the map showing the location of the nearest
restaurant, with the restaurant name and address as the pin's label.

From a coding standpoint, the way that this was implemented took a number of steps.
First, in the main storyboard of the application, we added a navigation bar button
to the top left of the MapView scene named "Find Food". We then linked this button
into the MapViewController.swift file to create an IBAction function called findFood().
Inside of the findFood function, we accomplished several things. The first and 
most significant task is that the function sends a request to the Yelp Fusion API
to recieve a JSON object with a list of all the restaurants close to the users current
application.

\subsection{Functions to be Used in Final Application}

\section{Challenges}

\subsection{Challenges in Tutorial}

\subsection{Challenges in Additions}

\end{document}
