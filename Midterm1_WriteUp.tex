\documentclass[conference]{IEEEtran}
\IEEEoverridecommandlockouts
% The preceding line is only needed to identify funding in the first footnote. If that is unneeded, please comment it out.
\usepackage{cite}
\usepackage{amsmath,amssymb,amsfonts}
\usepackage{algorithmic}
\usepackage{graphicx}
\usepackage{textcomp}
\usepackage{xcolor}
\def\BibTeX{{\rm B\kern-.05em{\sc i\kern-.025em b}\kern-.08em
    T\kern-.1667em\lower.7ex\hbox{E}\kern-.125emX}}
\begin{document}

\title{CS485G Midterm 1}

\author{\IEEEauthorblockN{Jonathan Moore}
\IEEEauthorblockA{\textit{Dept. of Electrical and Computer Engineering} \\
\textit{University of Kentucky}\\
Lexington, Kentucky \\
jdmo242@uky.edu}
\and
\IEEEauthorblockN{2\textsuperscript{nd} Given Name Surname}
\IEEEauthorblockA{\textit{dept. name of organization (of Aff.)} \\
\textit{name of organization (of Aff.)}\\
City, Country \\
email address}
\and
\IEEEauthorblockN{3\textsuperscript{rd} Given Name Surname}
\IEEEauthorblockA{\textit{dept. name of organization (of Aff.)} \\
\textit{name of organization (of Aff.)}\\
City, Country \\
email address}
}

\maketitle

\begin{abstract}
This project was to create an application that used the CoreLocation framework to track
and record the users location.
\end{abstract}

\section{Introduction}
The primary purpose of this tutorial was to gain experience using the CoreLocation
framework by creating an application that tracks and logs the users location. As an
extention of this primary purpose, the tutorial also offered a number of other opportunities
to learn about various frameworks and aspects of swift as well as experience in storyboarding
and working in Xcode. Through the app design process we were able to learn more about
table views using the UIKit table view controller feature. Using MapKit, we 
learned how to embed a map into the application as well as add custom points of 
interest to it. Another useful feature we gained experience with was the Codable
protocal, which provides an easy way to allow the app data to be encoded and decoded
for storage. We also learned how to use various features of the CoreLocation 
framework such as CLLocationCoordinate2D.
\section{Original Function}

\section{Functionality}

\subsection{Original Function}

\subsection{Added Functions}

\subsection{Functions to be Used in Final Application}

\section{Challenges}

\end{document}
